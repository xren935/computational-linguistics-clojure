\documentclass[10pt]{article}
 
\usepackage[margin=1in]{geometry} 
\usepackage{amsmath,amsthm,amssymb, graphicx, multicol, array}
\usepackage[T1]{fontenc}
\usepackage{textcomp}
\usepackage{url}
% use straight quotes in texttt environment, make 0 different from O
\usepackage[zerostyle=b,straightquotes,scaled=.93]{newtxtt}

\usepackage{listings}
\lstset{
 basicstyle=\ttfamily,
 columns=fullflexible,
 upquote,
 keepspaces,
 literate={*}{{\char42}}1
 {-}{{\char45}}1
}
\usepackage[short labels]{enumitem}

\usepackage{soul}  % for strike through (\st{})
\usepackage[dvipsnames,usenames,table]{xcolor} % for colors

\newcommand{\email}[1]{\href{mailto:#1}{\texttt{#1}}}
\newcommand{\PSnum}{2}

\begin{document}
 
\title{LING/COMP 445, LING 645\\Problem Set \PSnum}
\date{Due before 8:35 AM on Thursday, October 1, 2020}
\maketitle
There are several types of questions below. 
\begin{itemize}
\item
For questions involving answers in English or mathematics or a
combination of the two, put your answers to the question in an
\textbf{Answer} section like in the example below. You can find more
information about \LaTeX{} here \url{https://www.latex-project.org/}.

\item For programming questions,
please put your answers into a file called
\texttt{ps\PSnum-lastname-firstname.clj}. Be careful to follow the instructions
exactly and be sure that all of your function definitions use the
precise names, number of inputs and input types, and output types as
requested in each question.

For the code portion of the assignment, \textbf{it is crucial to submit a
standalone file that runs}. Before you submit \texttt{ps\PSnum-lastname-firstname.clj}, 
make sure that your code executes correctly without any errors 
when run at the command line by typing 
\texttt{clojure ps\PSnum-lastname-firtname.clj} at a terminal
prompt. We cannot grade any code that does not run correctly as a
standalone file, and if the preceding command produces an error,
the code portion of the assignment will receive a $0$.

To do the computational problems, we recommend that you install
Clojure on your local machine and write and debug the answers to each
problem in a local copy of \texttt{ps\PSnum-lastname-firstname.clj}. You can
find information about installing and using Clojure here
\url{https://clojure.org/}.
\end{itemize}
Once you have entered your answers, please compile your copy of this
\LaTeX{} into a PDF and submit 
\begin{enumerate}[(i),noitemsep]
\item
the compiled PDF renamed to
\texttt{ps\PSnum-lastname-firstname.pdf} 
\item
the raw \LaTeX{} file renamed to
\texttt{ps\PSnum-lastname-firstname.tex} and 
\item
your \texttt{ps\PSnum-lastname-firstname.clj}
\end{enumerate}
to the Problem Set \PSnum{} folder under `Assignments' on MyCourses.

\noindent\hrulefill %--------------------------------

\paragraph{Example Problem:}
This is an example question using some fake math like this
$L=\sum_0^{\infty} \mathcal{G} \delta_x$.

\paragraph{Example Answer:} Put your answer right under the question like
this $L=\sum_0^{\infty} \mathcal{G} \delta_x$.


\noindent\hrulefill %--------------------------------

\paragraph{Problem 1:}
Write a procedure called \texttt{abs} that takes in a number, and
computes the absolute value of the number. It should do this by
finding the square root of the square of the argument. (Note: you
should use the \texttt{Math/sqrt} procedure built in to Clojure (from Java), which
returns the square root of a number.)

\paragraph{Answer 1:} Please put your answer in \texttt{ps\PSnum-lastname-firstname.clj}.

\noindent\hrulefill %--------------------------------

\paragraph{Problem 2:}
In both of the following procedure definitions, there are one or more errors of
some kind. Explain what's wrong and why, and fix it:

\begin{lstlisting}
(defn take-square
  (* x x))
\end{lstlisting}

\begin{lstlisting}
(defn sum-of-squares [(take-square x) (take-square y)]
  (+ (take-square x) (take-square y)))
\end{lstlisting}

\paragraph{Answer 2:} Please put the fixed functions in
\texttt{ps\PSnum-lastname-firstname.clj} and describe what is wrong and why
here.
1. The first function take-square doesn't take an argument, x is therefore undefined in this function when we do (* x x).
To fix this, we can make the function take x as an input argument(so when we call it we pass the needed x(which is a number) to it).
2. The parentheses around the inputs are causing problem as they are binding take-square and x. We want to pass in take-square, x and y as 
inputs arguments and do not want the unsupported bindings inside the input arguments. 
We may also remove the second take-square and only use it once in the input argument surrounded by [], but this fix is not necessary. 
\noindent\hrulefill %--------------------------------

\paragraph{Problem 3:}
The expression \texttt{(+ 11 2)} has the value \texttt{13}. Write four
other different Clojure expressions whose values are also the number
\texttt{13}.  Using \texttt{def} name these expressions
\texttt{exp-13-1}, \texttt{exp-13-2}, \texttt{exp-13-3}, and
\texttt{exp-13-4}.

\paragraph{Answer 3:} Please put your answer in \texttt{ps\PSnum-lastname-firstname.clj} as
well as here.
(def exp-13-1 (+ 10 3))
(def exp-13-2 (+ (* 3 1) 10))
(def exp-13-3 (/ 26 2))
(def exp-13-4 (- 23 10))

\noindent\hrulefill %--------------------------------

\paragraph{Problem 4:}
  Write a procedure, called \texttt{third}, that selects the third
  element of a list. For example, given the list \texttt{'(4 5 6)}
  as its argument, \texttt{third} should return the number \texttt{6}.

\paragraph{Answer 4:} Please put your answer in \texttt{ps\PSnum-lastname-firstname.clj}.

\noindent\hrulefill %--------------------------------

\paragraph{Problem 5:}
Write a procedure, called \texttt{compose}, that takes two one-place
functions \texttt{f} and \texttt{g} as arguments. It should return a
new function, the composition of its input functions, which computes
\texttt{f(g(x))} when passed the argument \texttt{x}. For example, the
function \texttt{Math/sqrt} (built in to Clojure from Java) takes the
square root of a number, and the function \texttt{Math/abs} (likewise) takes the absolute value of a number. If we make
these functions Clojure native functions using \texttt{fn}, then
\texttt{((compose Math/sqrt Math/abs) -36)} should return \texttt{6},
because the square root of the absolute value of \texttt{-36} equals
\texttt{6}.

\begin{lstlisting}
(defn sqrt [x] (Math/sqrt x))
(defn abs [x] (Math/abs x))
((compose sqrt abs) -36)
\end{lstlisting}

\paragraph{Answer 5:} Please put your answer in \texttt{ps\PSnum-lastname-firstname.clj}.

\noindent\hrulefill %--------------------------------

\paragraph{Problem 6:}
  Write a procedure \texttt{first-two} that takes a list as its argument,
  returning a two element list containing the first two elements of
  the argument. For example, given the list \texttt{'(4 5 6)},
  \texttt{first-two} should return the list \texttt{'(4 5)}.

\paragraph{Answer 6:} Please put your answer in \texttt{ps\PSnum-lastname-firstname.clj}.

\noindent\hrulefill %--------------------------------

\paragraph{Problem 7:}
Write a procedure \texttt{remove-second} that takes a list, and
returns the same list with the second value removed. For example,
given \texttt{(list 3 1 4)}, \texttt{remove-second} should return
the list \texttt{(list 3 4)}.

\paragraph{Answer 7:} Please put your answer in \texttt{ps\PSnum-lastname-firstname.clj}.

\noindent\hrulefill %--------------------------------

\paragraph{Problem 8:}
  Write a procedure \texttt{add-to-end} that takes in two arguments: a
  list \texttt{l} and a value \texttt{x}. It should return a new list
  which is the same as \texttt{l}, except that it has \texttt{x} as
  its final element. For example, \texttt{(add-to-end (list 5 6 4) 0)}
  should return the list \texttt{(list 5 6 4 0)}.

\paragraph{Answer 8:} Please put your answer in \texttt{ps\PSnum-lastname-firstname.clj}.

\noindent\hrulefill %--------------------------------

\paragraph{Problem 9:}
  Write a procedure, called \texttt{reverse}, that takes in a list, and returns
  the reverse of the list. For example, if it takes in the list \texttt{'(a b c)}, it will output the list \texttt{'(c b a)}.

\paragraph{Answer 9:} Please put your answer in \texttt{ps\PSnum-lastname-firstname.clj}.

\noindent\hrulefill %--------------------------------

\paragraph{Problem 10:}
  Write a procedure, called \texttt{count-to-1}, that takes a positive
  integer \texttt{n}, and returns a list of the integers counting down
  from \texttt{n} to \texttt{1}. For example, given input \texttt{3},
  it will return the list \texttt{(list 3 2 1)}.

\paragraph{Answer 10:} Please put your answer in \texttt{ps\PSnum-lastname-firstname.clj}.

\noindent\hrulefill %--------------------------------

\paragraph{Problem 11:}
  Write a procedure, called \texttt{count-to-n}, that takes a positive
  integer \texttt{n}, and returns a list of the integers from
  \texttt{1} to \texttt{n}. For example, given input \texttt{3}, it
  will return the value of \texttt{(list 1 2 3)}. 
  
\paragraph{Hint:} 
  Use the procedures \texttt{reverse} 
  and \texttt{count-to-1} that you wrote in the previous problems.

\paragraph{Answer 11:} Please put your answer in \texttt{ps\PSnum-lastname-firstname.clj}.

\noindent\hrulefill %--------------------------------

\paragraph{Problem 12:}
  Write a procedure, called \texttt{get-max}, that takes a list of numbers,
  and returns the maximum value.

\paragraph{Answer 12:} Please put your answer in \texttt{ps\PSnum-lastname-firstname.clj}.

\noindent\hrulefill %--------------------------------

\paragraph{Problem 13:}
  Write a procedure, called \texttt{greater-than-five?}, that takes a
  list of numbers, and replaces each number with \texttt{true} if the
  number is greater than \texttt{5}, and \texttt{false} otherwise. For
  example, given input \texttt{(list 5 4 7)}, it will return the list
  \texttt{(list false false true)}. 

\paragraph{Hint:}
  Use the built in function \texttt{map}.

\paragraph{Answer 13:} Please put your answer in \texttt{ps\PSnum-lastname-firstname.clj}.

\noindent\hrulefill %--------------------------------

\paragraph{Problem 14:}
Write a procedure, called \texttt{concat-three}, that takes three
sequences (represented as lists), \texttt{x}, \texttt{y}, and
\texttt{z}, and returns the concatenation of the three sequences. For
example, given the sequences \texttt{(list 'a 'b)}, \texttt{(list 'b
  'c)}, and \texttt{(list 'd 'e)}, the procedure should return the list
\texttt{(list 'a 'b 'b 'c 'd 'e)}.

\paragraph{Answer 14:} Please put your answer in \texttt{ps\PSnum-lastname-firstname.clj}.

\noindent\hrulefill %--------------------------------

\paragraph{Problem 15:}
Write a procedure, called \texttt{sequence-to-power}, that takes a
sequence (represented as a list) \texttt{x}, and a positive integer
\texttt{n}, and returns the sequence \texttt{x}$^n$. For
example, given the sequence \texttt{(list 'a 'b)} and the number
\texttt{3}, the procedure should return 
\texttt{(list 'a 'b 'a 'b 'a 'b)}.

\paragraph{Answer 15:} Please put your answer in
\texttt{ps\PSnum-lastname-firstname.clj}.

\noindent\hrulefill %--------------------------------

\paragraph{Problem 16:}
Define $L$ as a language containing a single sequence,
$L=\{a\}$. Write a procedure \texttt{in-L?} that takes a sequence
(represented as a list), and decides if it is a member of the language
$L^*$. That is, given a sequence $x$, the procedure should return
\texttt{true} if and only if $x$ is a member of $L^*$, and
\texttt{false} otherwise.

\paragraph{Answer 16:} Please put your answer in
\texttt{ps\PSnum-lastname-firstname.clj}.

\noindent\hrulefill %--------------------------------

\paragraph{Problem 17:}
Let $A$ and $B$ be two distinct formal languages. We'll use
$(A\cdot B)$ to denote the concatenation of $A$ and $B$, in
that order. Find an example of languages $A$ and $B$ such that
$(A\cdot B)=(B\cdot A)$.

\paragraph{Answer 17:} Please put your answer here.

\noindent\hrulefill %--------------------------------

\paragraph{Problem 18:}
Let $A$ and $B$ be languages. Find an example of languages $A$ and $B$
such that $(A\cdot B)$ does not equal $(B\cdot A)$

\paragraph{Answer 18:} Please put your answer here.

\noindent\hrulefill %--------------------------------

\paragraph{Problem 19:}
Find an example of a language $L$ such that $L=L^2$,
i.e. $L=(L\cdot L)$.

\paragraph{Answer 19:} Please put your answer here.

\noindent\hrulefill %--------------------------------

\paragraph{Problem 20:}
Argue that the intersection of two languages $L$ and $L'$ is always
contained in $L$.

\paragraph{Answer 20:} Please put your answer here.

\noindent\hrulefill %--------------------------------

\paragraph{Problem 21:}
Let $L_1$, $L_2$, $L_3$, and $L_4$ be languages. Argue that the union
of Cartesian products $(L_1 \times L_3) \cup (L_2 \times L_4)$ is
always contained in the Cartesian product of unions
$(L_1 \cup L_2) \times (L_3 \cup L_4)$.

\paragraph{Answer 21:} Please put your answer here.

\noindent\hrulefill %--------------------------------

\paragraph{Problem 22:}
Let $L$ and $L'$ be finite languages. Show that the number of elements
in the Cartesian product $L \times L'$ is always equal to the number
of elements in $L' \times L$.

\paragraph{Answer 22:} Please put your answer here.

\noindent\hrulefill %--------------------------------

\paragraph{Problem 23:}

Suppose that the concatenation of a language L is equal to itself:
$(L\cdot L) = L$. Show that $L$ is either the empty set, 
the set $\{\epsilon\}$, or an infinite language.

\paragraph{Answer 23:} Please put your answer here.

\end{document}
